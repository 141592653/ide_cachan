\documentclass{article}
\usepackage[utf8]{inputenc}

\title{IDE\\\Large{Software Engineering Project Description}}
\author{Alice Rixte, K\'evin Le Run}
\date{September 2016}

\begin{document}

\maketitle

\section{General description}

This project aims to design an IDE (Integrated Development Environment) for
OCaml and possibly other languages. The goal is to organize the code in a
different way than the conventional file structure. The functions and classes
should be arranged in an intuitive way so as to facilitate software design.
However the creation and design of the text editor is not within the scope of
this project.

\begin{description}

    \item [Browser] The browser allows navigation through classes and functions
        in the same way than file browsers. With a function selected, the user
        can edit its implementation, its documentation, its tags, or its test
        functions and add categories within modules.  The user should also be
        able to interactively run the function or its tests in isolation.

    \item [Function/Method finder] Allows finding functions by name, by
        documentation keywords and by tags. Also allows finding functions by
        giving an example of the expected inputs and outputs (through the help
        of a simple language). As an example, giving 3 and 4 as inputs and
        asking for a function returning 7 could yield a function computing the
        addition of integers.

    \item [Dependency graph] Allows the user to see the code structure through
        a graph of class and module dependency.

    \item [Import/Export] Allows the user to switch between this IDE and a
        conventional project structure. It analyses the \texttt{.ml} files to
        extract the code structure.

    \item [Language Plugins] Allows adding more language support to the IDE
        with extensions that can be implemented separately from the main
        program, so that plugins can be produced by the community. It should be
        able to change the terminology of code structure (modules->packages,
        methods->functions) and the hierarchy while still preserving the
        file-system-like browsing and search features. It should redefine
        import and export features to accomodate to the new language.

\end{description}

\section{Detailed objectives}

\begin{itemize}
    \item [$\alpha$] Graphical User Interface (GUI)
    \item [$\alpha$] Saving, loading, exiting
    \item [$\alpha$] Browsing the structure
    \item [$\alpha$] Export
    \item [$\alpha$] Basic research
    \item [$\alpha$] External text editor integration

    \item [$\beta$] Import
    \item [$\beta$] Language plugins (with at least a Java plugin)
    \item [$\beta$] Advanced research
    \item [$\beta$] Interactive function execution
    \item [$\beta$] Compiler integration
    \item [$\beta$] Interactive testing system (run tests only on selected
        modules and functions)

    \item [??] Dependency graph
    \item [??] Debugger integration
    \item [??] Documentation generation
\end{itemize}

\section{Comments}

\texttt{// This is a comment}

\end{document}

