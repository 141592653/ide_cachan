\documentclass{beamer}

\usepackage[utf8]{inputenc}
\usepackage{default}
\usepackage{graphics}
\usepackage{listings}

\DeclareUnicodeCharacter{00A0}{ }

\title{ide}
\subtitle{Projet de génie logiciel}
\author{G. Le Bouder, K. Lerun, L. Prosperi, R. Zucchini}

\begin{document}
\maketitle

\begin{frame}
    \section{Outils}
    \frametitle{Gestion du code}

    \begin{itemize}
        \item Git
        \item Github
    \end{itemize}
\end{frame}

\begin{frame}
    \frametitle{Gestion de la communication}

    \begin{itemize}
        \item Framateam
        \item Github
        \item Mail
        \item La parole
    \end{itemize}
\end{frame}

\begin{frame}
    \frametitle{Technologies}

    \begin{itemize}
        \item Ocaml 4.04.0 (vive les warnings!!!!!)
        \item Compilation : ocamlbuild(.mlpack, \_tags)/make
        \item Tests : OUnit2
        \item Documentation : ocamldoc(.odocl)
    \end{itemize}
\end{frame}

\begin{frame}
    \frametitle{Core}

    \begin{itemize}
        \item Tag : Ajoute des informations \\
        Permet d'indiquer la nature du code (fonction, reférence) \\
        Pour l'instant des informations pour reconstituer l'ast à partir du c\_ast
        \item Gset : Devrait contenir des morceaux de code
    \end{itemize}
\end{frame}

\begin{frame}
    \frametitle{Miscs}

    c\_ast : arbre syntaxique extremement simple, avec des tags qui stocke des informations annexe propre à plugin
    \begin{itemize}
        \item exporter et importer le c\_ast.
        \item rechercher en fonction du nom des fonctions, signature, ..
        \item rechercher en fonction des tags.
    \end{itemize}
\end{frame}

\begin{frame}
    \frametitle{Miscs}
    \framesubtitle{C\_ast}

    \lstinputlisting[language=Caml]{part1.ml}
\end{frame}

\begin{frame}
    \frametitle{Miscs}
    \framesubtitle{C\_ast}

    \lstinputlisting[language=Caml]{part2.ml}
    \begin{itemize}
        \item Name : fonction1
        \item Header : x y
        \item Body : let fonction1 x y =x + y
        \item Children : Nil
    \end{itemize}
\end{frame}

\begin{frame}
    \frametitle{Miscs}
    \framesubtitle{Recherche}

    Module general\_table :\\ \ \\
    module de créer des tables de hachage qui pré-tri le c\_ast
    soit par nom soit par tag.\\
    On en peut extraire les nodes qui correspondent à un sous mot
    d'un nom ou d'un tag.
\end{frame}


\begin{frame}
\section{Le système de plugin}
\frametitle{}

 Objectifs :
\begin{itemize}
    \item Supporter un nouveau langage
    \item Supporter une nouvelle fonctionalitée
    \item Indépendance %compilation séparée
    \item Rôle : exporter/importer du code sources vers un \textit{c\_ast}
\end{itemize}

\end{frame}


\begin{frame}
    \frametitle{Un plugin}

    Classe Ocaml contenant :
    \begin{itemize}
        \item nom
        \item extensions concernées
        \item \textit{path\_to\_c\_ast}
        \item \textit{c\_ast\_to\_path}
        \item \textit{unitest}
    \end{itemize}
\end{frame}

\begin{frame}
    \frametitle{La gestion des plugins}

    Module \textit{Factory}
    \begin{itemize}
        \item Detecter les plugins
        \item Les charger en mémoire
    \end{itemize}
\end{frame}

\begin{frame}
    \frametitle{Autres fonctionnalités}

    \begin{itemize}
        \item Fonctions génériques pour le scan de dossier
    \end{itemize}
\end{frame}







\begin{frame}
    \section{Le plugin Ocaml}
    \frametitle{}

 Objectifs :
 \begin{itemize}
    \item Detecter les structures top-level
    \item Detecter la structure des modules, des classes
    \item Gérer les interactions ml/mli
 \end{itemize}

\end{frame}

\begin{frame}
    \includegraphics[width=\textwidth]{ocaml.png}
\end{frame}


\begin{frame}
\frametitle{Liste des structures Ocaml supportées - sans recursion}
    \begin{tabular}{ll}
         ouverture de module & open NameModule\\
         \pause
         déclaration de variable/constante & let x = ref ... \\
         \pause
         déclaration de fonction & \\
         \pause
         exception & exception Troll of int\\
         \pause
         déclaration de type & \\
         \pause
         contrainte de type & x: int list, x : int $->$ bool\\
         \pause
         les elements classe & \\
         & inherit avec/sans label\\
         & methode privée/public avec/sans contrainte\\
         & attribut avec/sans contrainte\\
         & avec/sans constructeur\\
    \end{tabular}
\end{frame}

\begin{frame}
\frametitle{Liste des structures Ocaml supportées - avec recursion}
    \begin{tabular}{ll}
        les classes & avec/sans arguments\\
        & avec/sans "self"\\
        & virtuelle ou non\\
        \pause
        les classes mutuellement recursives & \\
        \pause
        les classes avec contrainte & class ... :\\
        & object ... end = object ... end\\

        les modules & avec/sans signature\\
        \pause
        & mutuellement recursif ou non\\
        les signatures & \\
        \pause
        les foncteurs & \\
    \end{tabular}
\end{frame}

\begin{frame}
    \frametitle{Architecture de test}

    \begin{itemize}
        \item Fonction \textit{unitest} par sous-module
        \item Un sous-module \textit{Test} qui aggrége les tests
        \item Export du sous module \textit{Test}
    \end{itemize}
\end{frame}

\begin{frame}
    \frametitle{Politique de tests}

    \textbf{Ml vers AST temporaire}
    \begin{itemize}
        \item import : $>$ 1 test/structure (23 tests)
        \item export : $\ge$ 1 test/structure (14 tests)
    \end{itemize}

    \textbf{AST temporaire vers C\_ast}
    \begin{itemize}
        \item ~ 10 tests import/export
    \end{itemize}
\end{frame}

\begin{frame}
    \frametitle{Politique de documentation}

    \begin{itemize}
        \item Fonctions
            \begin{enumerate}
                \item Descriptig général
                \item Un commentaire par argument
                \item Description du type de sortie
                \item Descriptif des exceptions renvoyées
            \end{enumerate}
        \item Types
            \begin{enumerate}
                \item Descriptif général
                \item Descriptif des sous structures
                \item Exemples spécifiques
            \end{enumerate}
    \end{itemize}
\end{frame}

\begin{frame}
    \frametitle{Problèmes majeurs}

    \begin{itemize}
        \item Absence complète de documentation %reverse engeenering
            \uncover<2->{
                \lstinputlisting[language=Caml]{parsetree.ml}
            }
        \uncover<3->{
            \item Problème de design du coeur
        }
    \end{itemize}
\end{frame}

\end{document}
